%%%%%%%%%%%%%%%%%%%%%%%%%%%%%%%%%%%%%%%%%
% Beamer Presentation
% LaTeX Template
% Version 1.0 (10/11/12)
%
% This template has been downloaded from:
% http://www.LaTeXTemplates.com
%
% License:
% CC BY-NC-SA 3.0 (http://creativecommons.org/licenses/by-nc-sa/3.0/)
%
%%%%%%%%%%%%%%%%%%%%%%%%%%%%%%%%%%%%%%%%%

%----------------------------------------------------------------------------------------
%	PACKAGES AND THEMES
%----------------------------------------------------------------------------------------

\documentclass{beamer}

\mode<presentation> {

% The Beamer class comes with a number of default slide themes
% which change the colors and layouts of slides. Below this is a list
% of all the themes, uncomment each in turn to see what they look like.

%\usetheme{default}
%\usetheme{AnnArbor}
%\usetheme{Antibes}
%\usetheme{Bergen}
%\usetheme{Berkeley}
%\usetheme{Berlin}
%\usetheme{Boadilla}
%\usetheme{CambridgeUS}
%\usetheme{Copenhagen}
%\usetheme{Darmstadt}
%\usetheme{Dresden}
%\usetheme{Frankfurt}
%\usetheme{Goettingen}
%\usetheme{Hannover}
%\usetheme{Ilmenau}
%\usetheme{JuanLesPins}
%\usetheme{Luebeck}
\usetheme{Madrid}
%\usetheme{Malmoe}
%\usetheme{Marburg}
%\usetheme{Montpellier}
%\usetheme{PaloAlto}
%\usetheme{Pittsburgh}
%\usetheme{Rochester}
%\usetheme{Singapore}
%\usetheme{Szeged}
%\usetheme{Warsaw}

% As well as themes, the Beamer class has a number of color themes
% for any slide theme. Uncomment each of these in turn to see how it
% changes the colors of your current slide theme.

%\usecolortheme{albatross}
%\usecolortheme{beaver}
%\usecolortheme{beetle}
%\usecolortheme{crane}
\usecolortheme{dolphin}
%\usecolortheme{dove}
%\usecolortheme{fly}
%\usecolortheme{lily}
%\usecolortheme{orchid}
%\usecolortheme{rose}
%\usecolortheme{seagull}
%\usecolortheme{seahorse}
%\usecolortheme{whale}
%\usecolortheme{wolverine}

%\setbeamertemplate{footline} % To remove the footer line in all slides uncomment this line
\setbeamertemplate{footline}[page number] % To replace the footer line in all slides with a simple slide count uncomment this line

\setbeamertemplate{navigation symbols}{} % To remove the navigation symbols from the bottom of all slides uncomment this line
\setbeamertemplate{itemize items}[triangle]
}

\usepackage{graphicx} % Allows including images
\usepackage{booktabs} % Allows the use of \toprule, \midrule and \bottomrule in tables

\usepackage{longtable}
\usepackage{multirow}
\usepackage{dcolumn}
\usepackage{bigstrut}
\usepackage{dcolumn}
\newcolumntype{d}[1]{D{.}{.}{#1}}
%\usepackage[group-separator={.}]{siunitx}
\usepackage{xcolor,colortbl}
\definecolor{Gray}{gray}{0.85}
\usepackage{siunitx}
\usepackage{colortbl}
%\usepackage{arydshln}
\usepackage{adjustbox} % to use adjustbox command - change size of the table
\usepackage{breqn}

\usepackage{rotating}
\usepackage{epstopdf}
\usepackage{ragged2e}
\usepackage{bm}
%\usepackage[longnamesfirst,semicolon]{natbib}
%\bibliographystyle{agsm}
%\usepackage[citestyle=authoryear]{biblatex}
%\usepackage[backend=bibtex,style=authoryear, natbib]{biblatex}
\usepackage[backend=biber, style=authoryear, natbib, uniquename=false]{biblatex}
\bibliography{Lit_mod.bib}
\bibliography{Surprise_Lit.bib}

%----------------------------------------------------------------------------------------
%	TITLE PAGE
%----------------------------------------------------------------------------------------

\title[]{ \textbf{Machine Learning and Stock Market Returns}}  % The short title appears at the bottom of every slide, the full title is only on the title page


\author[Pavel Lesnevski]{Pavel Lesnevski }
\institute[University of Mannheim]{University of Mannheim}

\date{December 6, 2018} % Date, can be changed to a custom date

\begin{document}

\begin{frame}
\titlepage % Print the title page as the first slide
\end{frame}

%\begin{frame}
%\frametitle{Overview} % Table of contents slide, comment this block out to remove it
%\tableofcontents % Throughout your presentation, if you choose to use \section{} and \subsection{} commands, these will automatically be printed on this slide as an overview of your presentation
%\end{frame}

%----------------------------------------------------------------------------------------
%	PRESENTATION SLIDES
%----------------------------------------------------------------------------------------
%============================================================================

\section{Introduction}
\begin{frame}
	\frametitle{Motivation}
	%\framesubtitle{Bullet points}
	
	\begin{itemize}
		\item Controversial evidence on the ability of institutions to exploit mispricing-based anomalies  on the long side. {\scriptsize[Edelen, Ince, and Kadlec (2016), DeVault, Sias, and Starks (2016), Akbas et al. (2015)]}
		\item General profitability of short positions. {\scriptsize[\citet{desai_investigation_2002}, \citet{boehmer_which_2008}, \citet{diether_short-sale_2009}]} 
		\item Less attention is paid to short sellers' ability to exploit anomalies. Existing papers focus on the value and momentum anomalies. {\scriptsize[Dechow et al. (2001), Hanson and Sunderam (2014)]}
		\item I use aggregate short interest data and the mispricing score of Stambaugh, Yu, and Yuan (2015) to fill in the gap.
		\item I contribute to the literature by answering the following questions:
		\begin{itemize}

			\vspace*{0.1cm}
			\item Do short sellers exploit mispricing?
			\item Do they follow sentiment signal?
			\item Do they trade stocks with high limits to arbitrage strategically?
			\item Do they arbitrage mispricing away?
			\item Do short positions confirm predictions of theoretical models? 
		\end{itemize}
	\end{itemize}
\end{frame}
%============================================================================

\begin{frame}
		\frametitle{Related Papers}
	\begin{itemize}
\item Hanson and Sunderam (2014), RFS
	\begin{itemize}
\item Use time-variation in the cross-section of short interest to show that the significant growth of arbitrage capital exploiting momentum and value strategies is significantly related to the decrease in profitability of these strategies.
 	\end{itemize}
\item Hwang and Liu (2014), WP
	\begin{itemize}
\item Find that short sellers shy away from risky and prefer low-volatility, high-return  strategies  that  have  weak  correlations  with  other  strategies.
	\end{itemize}
\item Wu and Zhang (2015), WP
	\begin{itemize}
\item Show that short interest increasingly contains more return predictive information  beyond  anomalies, especially in more recent years.
	\end{itemize} 

 
% \item Other Related Papers: \citet{Akbas2015}, \citet{Jiao2016}, \citet{Drechsler2016}
	\end{itemize} 

\end{frame}
%============================================================================
\section{Data}
\begin{frame}
	\frametitle{Data}
\begin{itemize}
\item Sample period: January 1974 -- December 2010
\item Variables:
\begin{itemize}
\item Dependent: Log S\&P500 Index Returns

\item Independent variables:
\begin{itemize}
 \item Macro variables of \cite{Welch2008}
\end{itemize}
\end{itemize}
\end{frame}
%============================================================================
\begin{frame}
	\frametitle{Data}

	\begin{itemize}
		\item Mispricing measure (MISP) is the arithmetic average of ranking percentile for each of the 11 mispricing-based anomalies. \citep{Stambaugh2015}
		\begin{itemize}
			\item The alpha and the associated t-statistic are much higher for combined strategy compared to individual anomalies.
			\item Investor sentiment drives the dynamics of each of the 11 individual anomalies. \citep{Stambaugh2012a}
			\item Significant cross-sectional stock return predictability around the globe. \citep{Jacobs2016}
			\item Commonly used in the literature.
			\item Mispricing score is an ordinal value, i.e. it shows only that one stock is more or less overpriced than another stock in the cross-section, but does not show whether the mispricing increased or decreased over time. 
		\end{itemize}	
	\end{itemize}
\end{frame}  
%%========================================
%\begin{frame}
%	\frametitle{Descriptive Statistics}
%\begin{table}[htbp]
%  \centering
%  \footnotesize
%	  \resizebox{0.7\textwidth}{!}{
%	% Table generated by Excel2LaTeX from sheet 'summary'
\begin{tabular}{lccccccc}
\hline
\hline
\multicolumn{8}{l}{Panel A: Summary Statistics} \bigstrut\\
\hline
        &         &         & \multicolumn{5}{c}{Percentiles} \bigstrut\\
\cline{4-8}\multicolumn{1}{c}{Variable} & Mean    & SD      & 1st     & 10th    & Median  & 90th    & 99th \bigstrut\\
\hline
$SUSIR$ & 0.334   & 0.403   & -0.718  & -0.159  & 0.348   & 0.800   & 1.555 \bigstrut[t]\\
$SR$    & 0.027   & 0.023   & 0.003   & 0.005   & 0.018   & 0.060   & 0.094 \\
$DTC$   & 5.479   & 2.142   & 1.467   & 2.095   & 5.787   & 7.959   & 10.046 \\
$SR_{IO}$ & 0.058   & 0.024   & 0.023   & 0.032   & 0.050   & 0.091   & 0.129 \\
$MBETA$ & 1.056   & 0.063   & 0.865   & 0.990   & 1.048   & 1.143   & 1.163 \\
$SIZE$  & 3939.353 & 2230.917 & 896.804 & 1153.332 & 3571.290 & 6905.367 & 8159.658 \\
$BM$    & 0.670   & 0.123   & 0.449   & 0.504   & 0.665   & 0.840   & 0.965 \\
$RET\_RV$ & 0.012   & 0.050   & -0.117  & -0.046  & 0.017   & 0.070   & 0.120 \\
$RET\_MOM$ & 0.185   & 0.208   & -0.283  & -0.067  & 0.178   & 0.413   & 0.851 \\
$INV$   & 0.161   & 0.041   & 0.067   & 0.108   & 0.161   & 0.215   & 0.236 \\
$ROA$   & 0.053   & 0.016   & 0.009   & 0.032   & 0.054   & 0.066   & 0.091 \\
$MISP$  & 48.879  & 1.224   & 45.616  & 47.089  & 49.145  & 50.092  & 51.506 \\
$IVOLA$ & 0.019   & 0.004   & 0.014   & 0.015   & 0.018   & 0.024   & 0.032 \\
$HLSPREAD$ & 0.007   & 0.002   & 0.005   & 0.006   & 0.007   & 0.010   & 0.017 \\
$IO$    & 0.514   & 0.149   & 0.268   & 0.303   & 0.488   & 0.730   & 0.761 \bigstrut[b]\\
\hline
\end{tabular}%

%	\label{tab:summary}%
%	}
%\end{table}
%	\end{frame}
%%========================================
%============================================================================	
\begin{frame}
	\frametitle{Methodology}
	\begin{itemize}
		\item Panel regression adopted from \citet{Hanson2014}:
	\end{itemize}
		\begin{equation} \nonumber 
			SR_{i,t} = T_t + S_i+\bm{\beta}^{MISP\prime}  \bm{D}^{MISP}_{it-1} +\bm{\beta}^{BM\prime} \bm{D}^{BM}_{it-1}+\bm{\beta}^{Size\prime}  \bm{D}^{Size}_{it-1}  +\bm{\gamma}^\prime \bm{x}_{it-1} + \varepsilon_{i,t}
		\end{equation}

\begin{itemize}
\item[]
\begin{itemize}
	\item $T_t$ - month fixed effects
	\item $S_i$ - stock fixed effects
	\item $\bm{D}^{MISP}_{it}$ - vector of mispricing decile dummies
	\item $\bm{\beta}^{MISP\prime}_{it}$- vector of coefficients on mispricing dummies
	\item $\bm{x}_{it}$ - set of control variables (trading volume, institutional ownership, illiquidity, ivola, convertible debt dummy, dummies for stock exchanges, analyst coverage)
	\item Dummies for decile 5 are not included in the regression (reference decile)
	\item Decile 10 is the short side of the anomaly (overpriced), and decile 1  is the long side of the anomaly (underpriced)

\end{itemize}
\end{itemize}
\end{frame}  
%============================================================================

	\end{document}
